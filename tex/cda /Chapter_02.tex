%\documentclass[handout,xcolor=pdftex,dvipsnames,table]{beamer}
%\documentclass[xcolor=pdftex,dvipsnames,table]{beamer}
%\documentclass{beamer}
\documentclass[dvipdfmx, serif,handout]{beamer}
\input{../head2}
%\renewcommand{\baselinestretch}{1.5}
\renewcommand{\baselinestretch}{1}

%========================================================================
\begin{document}
\begin{frame}
  \titlepage
\end{frame}
%\begin{frame}{Outline}
%  \tableofcontents
  % You might wish to add the option [pausesections]
%\end{frame}
%========================================================================
\section{Chapter 2}
%========================================================================
\subsection{Probability Structure}
%========================================================================
\begin{frame}{\textsc{simple dataset}}
% latex table generated in R 3.2.0 by xtable 1.8-2 package
% Sat Apr 15 18:54:42 2017
\footnotesize
\begin{table}[ht]
\centering
\begin{tabular}{rllrrrll}
  \hline
id & Name & Gender & Age & HT & WT & Grade & Preference \\ 
  \hline
1 & Alfred & M &  14 & 69.00 & 112.50 & C & Pepsi \\ 
  2 & Alice & F &  13 & 56.50 & 84.00 & B & Pepsi \\ 
  3 & Barbara & F &  13 & 65.30 & 98.00 & B & Coke \\ 
  4 & Carol & F &  14 & 62.80 & 102.50 & B & Pepsi \\ 
  5 & Henry & M &  14 & 63.50 & 102.50 & A & Coke \\ 
  6 & James & M &  12 & 57.30 & 83.00 & A & Coke \\ 
  7 & Jane & F &  12 & 59.80 & 84.50 & B & Coke \\ 
  8 & Janet & F &  15 & 62.50 & 112.50 & B & Coffee \\ 
  9 & Jeffrey & M &  13 & 62.50 & 84.00 & B & Pepsi \\ 
  10 & John & M &  12 & 59.00 & 99.50 & C & Pepsi \\ 
   \hline
\end{tabular}
\end{table}


\end{frame}
%========================================================================
\begin{frame}{\textsc{frequency table from dataset}}

\begin{minipage}{0.4\hsize}
\footnotesize
% latex table generated in R 3.2.0 by xtable 1.8-2 package
% Sat Apr 15 19:01:01 2017
\begin{table}[ht]
\centering
\begin{tabular}{rr}
  \hline
Gender & freq  \\ 
  \hline
F &   5 \\ 
  M &   5 \\ 
   \hline
\end{tabular}
\end{table}
\end{minipage}
\begin{minipage}{0.4\hsize}
\begin{table}[ht]
\centering
\begin{tabular}{rr}
  \hline
Preference & freq \\ 
  \hline
Coffee &   1 \\ 
  Coke &   4 \\ 
  Pepsi &   5 \\ 
   \hline
\end{tabular}
\end{table}
\end{minipage}

% latex table generated in R 3.2.0 by xtable 1.8-2 package
% Sat Apr 15 19:05:33 2017
% latex table generated in R 3.2.0 by xtable 1.8-2 package
% Sat Apr 15 19:06:00 2017
\begin{table}[ht]
\centering
\begin{tabular}{rrrrr}
  \hline
 && \multicolumn{3}{c}{Preference}\\ 
 && Coffee & Coke & Pepsi \\ 
  \hline
Gender &F &   1 &   2 &   2 \\ 
 & M &   0 &   2 &   3 \\ 
   \hline
\end{tabular}
\end{table}



\end{frame}

%========================================================================
\begin{frame}{\textsc{$I \times J$ Contingency Table}}
\begin{itemize}
\item Let X and Y denote categorical variables
\item Contingency table: cells contain counts of outcomes for a sample.
\begin{itemize}
\item  Another name is cross-classification table. 
\end{itemize}
\end{itemize}

\centering
\begin{tabular}{cc|ccccc|c}
\hline
&		&\multicolumn{5}{c|}{$Y$}  & \\
&		&{$1$}  		&	{$\cdots$}		&	{$j$} 		&	{$\cdots$} &{$J$}  & \\\hline
$X$&$1$ 		& $n_{11}$ 	&	$\cdots$ 			&	$n_{1j}$ 	&	$\cdots$ 	&	$n_{1J}$&  $n_{1+}$ \\
&$\vdots$  	& $\vdots$ 	&		 		 	&	$\vdots$ 	&		 	&	$\vdots$&  $\vdots$ 	 \\
&{$i$} 		& {$n_{i1}$} 	&	{$\cdots$}	 	&	{$n_{ij}$} 	&	{$\cdots$}	&	{$n_{iJ}$}&  $n_{i+}$ \\
&$\vdots$ 	& $\vdots$ 	&		 		 	&	$\vdots$ 	&		 	&	$\vdots$&  $\vdots$ 	 \\
&I 		& $n_{I1}$ 	&	$\cdots$ 			&	$n_{Ij}$ 	&	$\cdots$ 	&	$n_{IJ}$&  $n_{I+}$ \\\hline
&		& $n_{+1}$ 	&	$\cdots$ 			&	$n_{+j}$ 	&	$\cdots$ 	&	$n_{+J}$&  $n$ \\
\end{tabular}
\end{frame}
%========================================================================
\begin{frame}{\textsc{Contingency Table}}

\bi
\item Hypothesis: aspirin intake reduces mortality from cardiovascular disease(心臓血管病). 
\item  contingency table of the relationship between aspirin use(X) and heart attacks(Y)
\item Study design
\begin{itemize}
\item 5-year randomized blind test study 
\item Subjects in the study did not know whether they were taking aspirin or a placebo. 
%Of the 11,034 physicians taking a placebo, 18 suffered fatal heart attacks over the course of the study, whereas of the 11,037 taking aspirin, 5 had fatal heart attacks.
\item Subjects took either one aspirin tablet or a placebo, every other day.
\end{itemize}
{\scriptsize
\begin{table}
\renewcommand{\arraystretch}{1.0}
\begin{tabular}{lccc} \\ \hline
& \multicolumn{3}{c}{Heart attacks} \\ \cline{2-4}
& Fatal & Nonfatal & No \\
& Attack & Attack & Attack \\ \hline
Placebo & 18 & 171 & 10,845 \\
Aspirin & 5 & 99 & 10,933 \\ \hline
\end{tabular}
\caption{\scriptsize Cross-Classification of Aspirin Use and Myocardial Infarction(心筋梗塞), TABLE 2.1 on page 37.}
\end{table}
}
\ei

\end{frame}

%========================================================================
%\begin{frame}{\textsc{$I \times J$ Contingency Table}}
%\begin{itemize}
%\item Let X and Y denote categorical response variables.
% \begin{itemize}
%\item X with I categories, Y with J categories
% \end{itemize}
%\item If we classify subjects on both variables, IJ possible combinations exist. 
%\item  The responses (X,Y) of a subject chosen randomly from some population have a probability distribution. 
%%\item A rectangular table having I rows for categories of X and J columns for categories of Y displays this distribution. 
%\item The cells 
%%of the table 
%represent the IJ possible outcomes.
%\item  When the cells contain frequency counts of outcomes for a sample, the table is called a contingency table
%\begin{itemize}
%\item  Another name is cross-classification table. 
%\end{itemize}
%\item  A contingency table with I rows and J columns is called an $I \times J$ table
%\item If $X$ has $I$ categories and $Y$ has $J$ categories, a contingency table has $I$ rows for $X$ and $J$ columns for $Y$ (i.e., $I \times J$ table).
%
%\end{itemize}
%
%\end{frame}
%========================================================================
\begin{frame}{\textsc{Contingency Tables and Their Distributions}}

\bi
\item The probability distribution $\{\pi_{ij}\}$ is the {\it joint distribution} of $X$ and $Y$, where
\begin{eqnarray*}
\pi_{ij} &=& \mbox{the prob. that $(X,Y)$ occurs in the cell}\\
&& \mbox{in row $i$ and column $j$.}
\end{eqnarray*}
\item The probability distribution $\{\pi_{i+}\}$ is the {\it marginal distribution} of $X$, where
$$\pi_{i+} = \sum_j \pi_{ij}$$
\item The probability distribution $\{\pi_{+j}\}$ is the {\it marginal distribution} of $Y$, where
$$\pi_{+j} = \sum_i \pi_{ij}$$
\ei

\end{frame}
%========================================================================
\begin{frame}{\textsc{$I \times J$ Contingency Table}}
\begin{itemize}
\item Let X and Y denote categorical variables
\item $\{\pi_{ij}\}$ is the {\it joint distribution} of $X$ and $Y$
\begin{itemize}
\item  Accordingly,  $\{\pi_{i+}\}$ and $\{\pi_{+j}\}$ are the marginal distribution of X and Y, respectively.
\end{itemize}
\end{itemize}

\centering
\begin{tabular}{cc|ccccc|c}
\hline
&		&\multicolumn{5}{c|}{$Y$}  & \\
&		&{$1$}  		&	{$\cdots$}		&	{$j$} 		&	{$\cdots$} &{$J$}  & \\\hline
$X$&$1$ 		& $\pi_{11}$ 	&	$\cdots$ 			&	$\pi_{1j}$ 	&	$\cdots$ 	&	$\pi_{1J}$&  $\pi_{1+}$ \\
&$\vdots$  	& $\vdots$ 	&		 		 	&	$\vdots$ 	&		 	&	$\vdots$&  $\vdots$ 	 \\
&{$i$} 		& {$\pi_{i1}$} 	&	{$\cdots$}	 	&	{$\pi_{ij}$} 	&	{$\cdots$}	&	{$\pi_{iJ}$}&  $\pi_{i+}$ \\
&$\vdots$ 	& $\vdots$ 	&		 		 	&	$\vdots$ 	&		 	&	$\vdots$&  $\vdots$ 	 \\
&I 		& $\pi_{I1}$ 	&	$\cdots$ 			&	$\pi_{Ij}$ 	&	$\cdots$ 	&	$\pi_{IJ}$&  $\pi_{I+}$ \\\hline
&		& $\pi_{+1}$ 	&	$\cdots$ 			&	$\pi_{+j}$ 	&	$\cdots$ 	&	$\pi_{+J}$&  1\\
\end{tabular}
\end{frame}

%========================================================================
\begin{frame}{\textsc{Conditional Distributions}}

\bi
\item The probabilities $\{\pi_{1 \mid i}, \ldots, \pi_{J \mid i}\}$ form the {\it conditional distribution} of $Y$ at category $i$ of $X$, where
\begin{eqnarray*}
\pi_{j \mid i} &=& \mbox{the prob. of classification in column $j$ of $Y$} \\
&& \mbox{given that a subject is classified in row $i$ of $X$.}
\end{eqnarray*}
\centering
\begin{tabular}{cc|ccccc|c}
\hline
&		&\multicolumn{5}{c|}{$Y$}  & \\
&		&{$1$}  		&	{$\cdots$}		&	{$j$} 		&	{$\cdots$} &{$J$}  & \\\hline
$X$&$1$ 		& $\pi_{1|1}$ 	&	$\cdots$ 			&	$\pi_{j|1}$ 	&	$\cdots$ 	&	$\pi_{J|1}$&  1\\
&$\vdots$  	& $\vdots$ 	&		 		 	&	$\vdots$ 	&		 	&	$\vdots$&  $\vdots$ 	 \\
&{$i$} 		& {$\pi_{1|i}$} 	&	{$\cdots$}	 	&	{$\pi_{j|i}$} 	&	{$\cdots$}	&	{$\pi_{J|i}$}&  1\\
&$\vdots$ 	& $\vdots$ 	&		 		 	&	$\vdots$ 	&		 	&	$\vdots$&  $\vdots$ 	 \\
&I 		& $\pi_{1|I}$ 	&	$\cdots$ 			&	$\pi_{j|I}$ 	&	$\cdots$ 	&	$\pi_{J|I}$&  1 \\\hline
&		& $\pi_{1+}$ 	&	$\cdots$ 			&	$\pi_{j+}$ 	&	$\cdots$ 	&	$\pi_{J+}$&  1\\
\end{tabular}

\ei

\end{frame}
%%========================================================================
\begin{frame}{\textsc{Sensitivity and Specificity}}

\bi
\item {\bf{\em Sensitivity}}: the conditional probability that the diagnostic test is positive given that the subject has the disease.
\item {\bf{\em Specificity}}: the conditional probability that the test is negative given that the subject does not have the disease.
{\scriptsize
\begin{table}
\renewcommand{\arraystretch}{1.0}
\begin{tabular}{lccc} \\ \hline
Breast & \multicolumn{2}{c}{Diagnosis of Test} \\ \cline{2-3}
Cancer & Positive & Negative & Total \\ \hline
Yes & 0.82 & 0.18 & 1.0 \\
No & 0.01 & 0.99 & 1.0 \\ \hline
\end{tabular}
\caption{\scriptsize Estimated Conditional Distributions for Breast Cancer Diagnosis. TABLE 2.2 on page38.}
\end{table}
}
{\scriptsize
\begin{table}
\renewcommand{\arraystretch}{1.0}
\begin{tabular}{lccc} \\ \hline
Breast & \multicolumn{2}{c}{Diagnosis of Test} \\ \cline{2-3}
Cancer & Positive & Negative & Total \\ \hline
Yes & $\pi_{+|Yes}$ & $\pi_{-|Yes}$ & 1.0 \\
No & $\pi_{+|No}$ & $\pi_{-|No}$ & 1.0 \\ \hline
\end{tabular}
\caption{\scriptsize Prob. of cell structure}
\end{table}
}


\ei

\end{frame}
%%========================================================================
\begin{frame}{\textsc{Independence of Categorical Variables}}

\bi
\item Two categorical variables $X$ and $Y$ are independent if
\begin{eqnarray*}
\pi_{ij} &=& \pi_{i+}\pi_{+j}\;\;\; \mbox{for}\;\;\; i=1,\ldots, I\;\;\;\mbox{and}\;\;\; j=1,\ldots,J. \\
\Leftrightarrow \pi_{j \mid i} &=& \pi_{ij}/\pi_{i+}=\pi_{+j}\;\;\; \mbox{for}\;\;\; i=1,\ldots, I.
\end{eqnarray*}
\item Independence is often referred to as {\bf{\em homogeneity}} of the conditional distributions.
\item Sample distributions use similar notation, with $p$ or $\hat{\pi}$ in place of $\pi$.
\ei

\end{frame}
%========================================================================
\begin{frame}{\textsc{Poisson Sampling}}

\bi
\item Poisson sampling treats cell counts $\{Y_{ij}\}$ as independent Poisson random variables with parameter $\{\mu_{ij}\}$.
$$\{Y_{ij}\} \stackrel{indep.}{\sim} Poisson(\mu_{ij})$$
\item The joint p.m.f. for $\{n_{ij}\}$ is
$$\prod_i \prod_jP(Y_{ij}=n_{ij}) = \prod_i \prod_j \frac{\exp(-\mu_{ij})\mu_{ij}^{n_{ij}}}{n_{ij}!}$$
\ei

\end{frame}
%========================================================================
\begin{frame}{\textsc{Multinomial Sampling}}

\bi
\item The total sample size $n$ is fixed but the row and column totals, $n_{i+}$ and $n_{+j}$, are not fixed:
{\small
$$P(Y_{ij}=n_{ij}) = \frac{n!}{\prod_i \prod_j n_{ij}!} \prod_i \prod_j \pi_{ij}^{n_{ij}}.$$
}
\item When row totals $\{n_{i+}\}$ are fixed:
{\small
$$P(Y_{ij}=n_{ij}) = \frac{n_{i+}!}{\prod_j n_{ij}!} \prod_j \pi_{j \mid i}^{n_{ij}}.$$
}
This sampling scheme is {\it independent} (or {\it product}) {\it multinomial sampling}.
\ei

\end{frame}
%========================================================================
\begin{frame}{\textsc{Hypergeometric Sampling}}

\bi
\item When both row and column margins are fixed in $2\times 2$ table, the {\it hypergeometric sampling} applies:
{\small
$$P(n_{11}=t)=\frac{\left( \begin{array}{c} n_{1+} \\ t \end{array} \right) 
\left( \begin{array}{c} n_{2+} \\ n_{+1}-t \end{array} \right)}{\left( \begin{array}{c} n \\ n_{+1} \end{array} \right)}$$
}
\ei

This will be discussed in Section 3.5.1. 

%{\footnotesize
%\begin{table}
%\begin{tabular}{lcc} \hline
%%Seat-Belt Use & Frailty & Nonfrailty \\ \hline
%$n_{11}$&$n_{12}$& \\
%$n_{21}$&$n_{22}$& \\ \hline
%$n_{+1}$&$n_{+2}$& n\\ \hline
%\end{tabular}
%\end{table}
%}

\end{frame}
%========================================================================
\begin{frame}{\textsc{Seat Belt Example}}

{\footnotesize
\begin{table}
\begin{tabular}{lcc} \hline
& \multicolumn{2}{c}{Result of Crash} \\ \cline{2-3}
Seat-Belt Use & Frailty & Nonfrailty \\ \hline
Yes \\
No \\ \hline
\end{tabular}
\caption{\footnotesize Seat-Belt Use and Results of Automobile Crashes. TABLE 2.4 on page 41.}
\end{table}
}

\end{frame}
%========================================================================
\begin{frame}{\textsc{Types of Study}}

\bi
\item Retrospective study (case-control study)
\item Prospective study
\begin{enumerate}
\item clinical trials
\item cohort study
\end{enumerate}
\item Cross-sectional study
\ei

\end{frame}
%========================================================================
\begin{frame}{\textsc{Types of Study}}

{\footnotesize
\begin{table}
\begin{tabular}{lrr} \hline
& \multicolumn{2}{c}{Lung Cancer} \\ \cline{2-3}
Smoker & Cases & Control \\ \hline
Yes & 688 & 650 \\
No  & 21 & 59 \\ \hline
\hspace{.2cm} Total & 709 & 709 \\ \hline
\end{tabular}
\caption{\footnotesize Cross-Classification of Smoking by Lung Cancer. TABLE 2.5 on page 42.}
\end{table}
}

\end{frame}
%========================================================================
\subsection{Comparing Proportions}
%========================================================================
\begin{frame}{\textsc{Comparing Two Proportions}}

\vspace{-.3cm}
\begin{center}
{\footnotesize
\renewcommand{\arraystretch}{.8}
\begin{tabular}{cccc|} \\
&& \multicolumn{2}{c}{$Y$} \\ 
&  & S & \multicolumn{1}{c}{F} \\ \cline{3-4}
& \multicolumn{1}{c|}{1} & $\pi_1$ & $1-\pi_1$ \\
\raisebox{1.5ex}[0pt]{$X$} & \multicolumn{1}{c|}{2} & $\pi_2$ & $1-\pi_2$ \\ \cline{3-4}
\end{tabular}
}
\end{center}
\benu
\item Difference:
\bi
\item $D=\pi_1-\pi_2$
\item $D=0$ $\Rightarrow$ $Y$ is independent of the row classification.
\ei
\item Relative risk:
\bi
\item $RR = \pi_1/\pi_2$
\item $RR=1$ $\Rightarrow$ independence
\ei
\eenu

\end{frame}
%========================================================================
\begin{frame}{\textsc{Comparing Two Proportions}}

\benu
\item[3.] Odds Ratio:
\bi
\item $\theta = \frac{\Omega_1}{\Omega_2}=\frac{\pi_1/(1-\pi_1)}{\pi_2/(1-\pi_2)}$
\item For joint distributions with cell probabilities $\{\pi_{ij}\}$,
$$\theta=\frac{\pi_{11}/\pi_{12}}{\pi_{21}/\pi_{22}}=\frac{\pi_{11}\pi_{22}}{\pi_{12}\pi_{21}}$$
\item $\theta=1$ $\Rightarrow$ independence
\item The sample odds ratio is
$$\hat{\theta} = \frac{n_{11}n_{22}}{n_{12}n_{21}}$$
\item $\theta=RR \left(\frac{1-\pi_2}{1-\pi_1} \right)$
\ei
\eenu

\end{frame}
%========================================================================
\begin{frame}{\textsc{Aspirin and Heart Attacks Revisited}}

\begin{itemize}
\item Combining "fatal" and "non-fatal" category as "yes-Attack".
\end{itemize}
{\scriptsize
\begin{table}
\renewcommand{\arraystretch}{1.0}
\begin{tabular}{lcc} \\ \hline
& \multicolumn{2}{c}{Heart attack} \\ \cline{2-3}
& Yes & No \\
& Attack & Attack \\ \hline
Placebo & 189  & 10,845 \\
Aspirin & 104  &10,933 \\ \hline
\end{tabular}
\caption{\scriptsize Cross-Classification of Aspirin Use and Myocardial Infarction(心筋梗塞), TABLE 2.1 on page 37.}
\end{table}
}

\bi
\item The sample difference of proportions 0.0077
\item The relative risk is 1.82
\item The sample odds ratio is 1.83
\ei

\end{frame}
%========================================================================
\begin{frame}{\textsc{Odds Ratio}}

{\scriptsize
\begin{table}
\renewcommand{\arraystretch}{1.0}
\begin{tabular}{lcc} \\ \hline
 & \multicolumn{2}{c}{Lung Cancer} \\ \cline{2-3}
Smoker & Cases & Controls \\ \hline
Yes & 688 & 650 \\
No & 21 & 59 \\ \hline
\end{tabular}
\caption{\scriptsize Cross-Classification of Smoking by Lung Cancer. TABLE 2.5 on page 42.}
\end{table}
}

\bi
\item The odds ratio is equally valid for prospective, retrospective, or cross-sectional sampling designs.
\ei

\end{frame}
%========================================================================
\subsection{Stratified Tables}
%%========================================================================
\begin{frame}{\textsc{Partial Association in Stratified $2\times 2$ Tables}}

\bi
\item In studying the effect of $X$ and $Y$, one should control any covariate that can influence that relationship.
\item In this section we discuss the analysis of the association between categorical variables $X$ and $Y$ while controlling for a possibly confounding variable $Z$.
\ei

\end{frame}
%========================================================================
\begin{frame}{\textsc{Marginal Table}}

\bi
\item Consider Death Penalty Example:
\vspace{.5cm}
{\scriptsize
\begin{table} %\small
\begin{tabular}{cccc} \hline
Defendant's & \multicolumn{2}{c}{Death Penalty} (Y)& Percent \\ \cline{2-3}
{Race} (X)& {\bf{\em Yes}} & {\bf{\em No}} & Yes \\ \hline
{\bf{\em White}} & 53 & 430 & 11.0 \\
{\bf{\em Black}} & 15 & 176 & 7.9 \\ \hline
\end{tabular}
\caption{\scriptsize Death Penalty Verdict by Defendant's Race. TABLE 2.6 on page 48.}
\end{table}
}
\ei

\end{frame}
%========================================================================
\begin{frame}{\textsc{Partial Table}}

\bi
\item But if you adjust for another classification, such as {\em victim's race},
\vspace{.5cm}
{\scriptsize
%\renewcommand{\arraystretch}{1.2}
\begin{table} 
{\tabcolsep=0.03in
\begin{tabular}{ccccccccc} \hline
\multicolumn{4}{c}{{Victim's race}} \\
\multicolumn{4}{c}{\bf{\em White}} & & \multicolumn{4}{c}{\bf{\em Black}} \\ \cline{1-4} \cline{6-9} 
Defendant's & \multicolumn{2}{c}{Death Penalty} & Percent & & Defendant's &
\multicolumn{2}{c}{Death Penalty} & Percent \\ \cline{2-3} \cline{7-8}   
{Race} & {\bf{\em Yes}} & {\bf{\em No}} & Yes & & {Race} & {\bf{\em Yes}} & {\bf{\em No}} & Yes \\ \cline{1-4} \cline{6-9} 
{\bf{\em White}} & 53 & 414 & 11.3 & & {\bf{\em White}} & 0 & 16 & 0.0 \\
{\bf{\em Black}} & 11 & 37 & 22.9 & & {\bf{\em Black}} & 4 & 139 & 2.8 \\ \hline
\end{tabular}
\caption{\scriptsize Death Penalty Verdict by Defendant's Race and Victim's Race. TABLE 2.6 on page 48.}
}
\end{table}
}
\ei

\end{frame}
%========================================================================
\begin{frame}{\textsc{Conditional and Marginal Odds Ratio}}

\bi
\item $XY$ conditional odds ratio:
$$\theta_{XY(k)} = \frac{\mu_{11k}\mu_{22k}}{\mu_{12k}\mu_{21k}}$$
\item $XY$ marginal odds ratio:
$$\theta_{XY} = \frac{\mu_{11+}\mu_{22+}}{\mu_{12+}\mu_{21+}}$$
\ei

\end{frame}
%========================================================================
\begin{frame}{\textsc{Marginal versus Conditional Independence}}

\bi
\item If $X$ and $Y$ are independent in partial table $k$,
\begin{eqnarray}
P(Y=j \mid X=i, Z=k) = P(Y=j \mid Z=k), \;\; \mbox{for all}\;\; i,j.
\label{eq1}
\end{eqnarray}
\item $X$ and $Y$ are conditionally independent when (\ref{eq1}) holds for all $k$.
\ei

\end{frame}
%========================================================================
\begin{frame}{\textsc{Marginal versus Conditional Independence}}

{\scriptsize
\begin{table} %\small
\begin{tabular}{lcrr} \hline
&& \multicolumn{2}{c}{Response}(Y) \\ \cline{3-4}
Clinic(Z) & Treatment(X) & Success & Failure \\ \hline
1 & A & 18 & 12 \\
  & B & 12 & 8 \\
2 & A & 2 & 8 \\
  & B & 8 & 32  \\ \cline{3-4}
  \hspace{.2cm}Total & A & 20 &20 \\
                     & B& 20 & 40 \\ \hline
\end{tabular}
\caption{\scriptsize Expected Frequencies Showing That Conditional Independence Does Not Imply Marginal Indepepndence. TABLE 2.7 on page 53}
\end{table}
}

\end{frame}
%========================================================================
\begin{frame}{\textsc{Homogeneous Association}}

\bi
\item A $2 \times 2 \times K$ table has {\bf{\em homogeneous $XY$ association}} when
$$\theta_{XY(1)} = \theta_{XY(2)} = \cdots =\theta_{XY(K)}.$$
\ei

\end{frame}
%========================================================================
\begin{frame}{\textsc{Ordinal Trends: Concordant and Discordant Pairs}}
% latex table generated in R 3.2.0 by xtable 1.8-2 package
% Sat Apr 15 19:30:29 2017

\begin{itemize}
\item When X and Y are ordinal, a monotone trend association is common. 
\item As variable X $\uparrow$ $\Rightarrow$ variable Y tend to $\uparrow$ or $\downarrow$.
\item For instance, perhaps job satisfaction(Y) tends to increase as income(X) does. 
\end{itemize}

{\scriptsize
\begin{table} %\small
\begin{tabular}{rrrrr}
  \hline
 & \multicolumn{4}{c}{Job satisfaction(Y)}\\ 
Income(X) & VD & LD & MS & VS \\ 
  \hline
$<$15000 & 1 & 3 & 10 & 6 \\ 
  15000-25000 & 2 & 3 & 10 & 7 \\ 
  25000-40000 & 1 & 6 & 14 & 12 \\ 
  $>$40000 & 0 & 1 & 9 & 11 \\ 
   \hline
\end{tabular}
\caption{\scriptsize Cross-Classification of Job Satisfaction by Income. TABLE 2.8 on page 57.}
\end{table}
}

\end{frame}
%========================================================================
\begin{frame}{\textsc{Ordinal Trends: Concordant and Discordant Pairs}}

\begin{itemize}
\item Association  measures are based on pair of subjects as concordant or discordant. 
\item A pair of concordant(C): the subject ranked higher on X also ranks higher on Y. 

\item A pair of discordant(D): the subject ranking higher on X ranks lower on Y. 


\end{itemize}

\end{frame}
%========================================================================
\begin{frame}{\textsc{Ordinal Trends: Concordant and Discordant Pairs}}
\begin{itemize}
\item Consider joint cell probability distribution $\pi_{ij}$
\end{itemize}

\begin{itemize}
\item the probabilities of concordance
$\Pi_{c} = 2 \sum_{i} \sum_{j} \pi_{ij}(\sum_{h>i} \sum_{k>j} \pi_{hk})$
\item the probabilities of disconcordance
$\Pi_{d} = 2 \sum_{i} \sum_{j} \pi_{ij}(\sum_{h>i} \sum_{k<j} \pi_{hk})$
\item association measures for ordinal variables 
$$\Pi_c - \Pi_d$$
\end{itemize}


{\scriptsize
\begin{table} %\small
\begin{tabular}{rrrrr}
  \hline
 & \multicolumn{4}{c}{Job satisfaction(Y)}\\ 
Income(X) & VD & LD & MS & VS \\ 
  \hline
$<$15000 	& $\pi_{11}$ 		& $\pi_{12}$ 	& $\pi_{13}$ 	 		& $\pi_{14}$ \\ 
  15000-25000 	& $\pi_{21}$ 		& $\pi_{22}$ 	& $\pi_{23}$  		& $\pi_{24}$ \\ 
  25000-40000 	& $\pi_{31}$ 		& $\pi_{32}$ 	& $\pi_{33}$  		& $\pi_{34}$  \\ 
  $>$40000 	& $\pi_{41}$ 		& $\pi_{42}$ 	& $\pi_{43}$ 			& $\pi_{44}$  \\ 
   \hline
\end{tabular}
\caption{\scriptsize Cross-Classification of Job Satisfaction by Income}
\end{table}
}


\end{frame}
%========================================================================
\begin{frame}{\textsc{Ordinal Measure of Association: Gamma}}
\begin{itemize}
\item Given that a pair is untied on both variables, 
\begin{itemize}
\item cond. prob. of  concordance : $\frac{\Pi_c}{\Pi_c + \Pi_d}$
\item cond. prob. of disconcordance : $\frac{\Pi_d}{\Pi_c + \Pi_d}$
\end{itemize}
\item $\gamma$ is  difference between these probabilities
$$\gamma=\frac{\Pi_c-\Pi_d}{\Pi_c + \Pi_d}$$
$$-1<\gamma<1$$
\end{itemize}

\end{frame}
%========================================================================
\end{document}
