\documentclass{jarticle}
\usepackage{otf}

\usepackage{mathptmx}
\usepackage{amsmath}





\begin{document}


\section{aaa}

\textbf{日本語}や{\Huge 日本語}と



\( x \)

$x$

\centering \Huge

本章では,個個々の親測デー個々が独立に戦一の正規分布に位うと役定できる場合の異常檢知の手法を学びます。

「テリング理論」として知られる多\UTF{53D8}量解旅における外れ\UTF{503C}檢出手法がその中心です。いわば異常柄知の古典理論といえます。その\UTF{6B77}史の長さに比例して理論の奧も深いので,初荒の際は「*」の付いた節を飛ばして記むとよいでしょう。


2.1異常椅知手順の流れ

出出問題を念頭に,1.3節で述べた異常\UTF{6986}知の手順を改めてまとめます。
0)準備:まず,異常檢知を行うためにはデー個々の準備が必要です。
ここでは。\UTF{5218}象とする系に現測を施した結果。M次元の\UTF{9510}測\UTF{503C}が$N$個
手元にあると役定します。デー個々をまとめてのという記号で表し,この
中には異常な箱測\UTF{503C}が含まれていないか,含まれていたとしてもその影
響は無視できると役定します。

$$
\mathcal{D}=\left\{\boldsymbol{x}^{(1)},\boldsymbol{x}^{(2)},\ldots,\boldsymbol{x}^{(N)}\right\}
$$


呈鳴檢失口
%本章では,個個々の親測デー個々が独立に戦一の正規分布に位うと役定できる場合の異常檢知の手法を学びます。「テリング理論」として知られる多\UTF{53D8}量解旅における外れ\UTF{503C}檢出手法がその中心です。いわば異常柄知の古典理論といえます。その\UTF{6B77}史の長さに比例して理論の奧も深いので,初荒の際は「*」の付いた節を飛ばして記むとよいでしょう。
2.1異常椅知手順の流れ
外れ\UTF{503C}検出出問題を念頭に,1.3節で述べた異常\UTF{6986}知の手順を改めてまとめます。
0)準備:まず,異常檢知を行うためにはデー個々の準備が必要です。
ここでは。\UTF{5218}象とする系に現測を施した結果。M次元の\UTF{9510}測\UTF{503C}が$N$個
手元にあると役定します。デー個々をまとめてのという記号で表し,この
中には異常な箱測\UTF{503C}が含まれていないか,含まれていたとしてもその影
響は無視できると役定します。
$$
%\mathcal{D}=\left\{\boldsymbol{x}^{(1)},\boldsymbol{x}^{(2)},\ldots,\boldsymbol{x}^{(N)}\right\}
$$
1)ステップ1(分布推定):ここでは,デー個々の性質に忘じた適切な確率
分布のモデルを役定します。一般に確率分布はデー個々から定めるべきパ
ラメ個々ーをいくつか含みますので,それをまとめて$\theta$という記号で表し
%ておきます。典型的には;分布推定の問題とは,$p(\boldsymbol{x}\mid\boldsymbol{\theta})$における未知
%パラメ個々ー$\boldsymbol{\theta}$を、$\mathcal{D}$から決める問題です。
2)ステップ2(異常度の定義):$\quad$未知パラメ個々ーをデー個々から決めるなど



1)ステップ1(分布推定):ここでは,デー個々の性質に忘じた適切な確率
分布のモデルを役定します。一般に確率分布はデー個々から定めるべきパ
ラメ個々ーをいくつか含みますので,それをまとめて$\theta$という記号で表し
%ておきます。典型的には;分布推定の問題とは,$p(\boldsymbol{x}\mid\boldsymbol{\theta})$における未知
%パラメ個々ー$\boldsymbol{\theta}$を、$\mathcal{D}$から決める問題です。
2)ステップ2(異常度の定義):$\quad$未知パラメ個々ーをデー個々から決めるなど


\end{document}


(
/usr/local/texlive/2018/texmf-dist/tex/platex/jsclasses/

